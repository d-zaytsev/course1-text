\titleformat{\chapter}[display]
  {\normalfont\bfseries}{}{0pt}{\Huge}
\chapter{Обзор}
\section{Анализ существующих решений}
Перед началом работы над приложением был проведён анализ существующих решений. Целью было выявление их преимуществ и недостатков, а также отбор интересных идей, которые могли бы быть полезными для пользователя. \par
Для оценки были собраны наиболее информативные отзывы от пользователей на несколько самых популярных приложений. Ниже я приведу результаты оценки каждого из приложений.
% ---
\subsection*{Absolute ear\cite{Apps1}}
\addcontentsline{toc}{subsection}{Absolute ear}
\begin{minipage}[t]{0.45\textwidth}
\textbf{Достоинства:}
\begin{itemize}
  \item[+] Есть раздел с теорией, интегрированный в обучение
  \item[+] Большое количество упражнений, развитие самых разных навыков
  \item[+] Оценка и поощрение достижений пользователя, дополнительная мотивация к зантиям
  \item[+] Возможность просмотра статистики и выявления слабых мест
\end{itemize}
\end{minipage}
\hfill
\begin{minipage}[t]{0.45\textwidth}
\textbf{Недостатки:}
\begin{itemize}
  \item[-] Возможны баги, при которых результат пользователя оценивается некорректно
  \item[-] Интерфейс не уведомляет пользователя об ошибках
  \item[-] Интерфейс ориентирован только на знающих терминологию людей
  \item[-] Почти полностью отсутствует возможность настройки упражнений
\end{itemize}
\end{minipage}
% ---
\subsection*{Functional Ear Trainer\cite{Apps2}}
\addcontentsline{toc}{subsection}{Functional Ear Trainer}
\begin{minipage}[t]{0.45\textwidth}
\textbf{Достоинства:}
\begin{itemize}
  \item[+] Есть режим с полной настройкой упражнения
  \item[+] Интерфейс адаптирован под всех пользователей (разные форматы отображения информации)
  \item[+] Большое количество настроек как интерфейса, так и всего остального
\end{itemize}
\end{minipage}
\hfill
\begin{minipage}[t]{0.45\textwidth}
\textbf{Недостатки:}
\begin{itemize}
  \item[-] Музыкальные обозначения не подходят для всех пользователей. В России они другие
  \item[-] Упражнения не учитывают предыдущие попытки пользователя, возможны повторения
  \item[-] Из-за платной подписки урезан основной контент
\end{itemize}
\end{minipage}
% ---
\subsection*{My Ear Trainer\cite{Apps3}}
\addcontentsline{toc}{subsection}{My Ear Trainer}
\begin{minipage}[t]{0.45\textwidth}
\textbf{Достоинства:}
\begin{itemize}
  \item[+] Уникальный режим - подбор целой мелодии на слух
  \item[+] Большое разнообразие упражнений и выбор сложности для каждого
  \item[+] Есть курсы с введением в теорию	
\end{itemize}
\end{minipage}
\hfill
\begin{minipage}[t]{0.45\textwidth}
\textbf{Недостатки:}
\begin{itemize}
  \item[-] Почти нет настроек интерфейса под разных пользователей
  \item[-] Нет поддержки некоторых языков и их обозначений
\end{itemize}
\end{minipage}
\section{Требования к приложению}
Исходя из результатов проведённого анализа, были сформулированы \textbf{требования}. Обучающее приложение должно: 
\begin{enumerate}
\item запускаться на телефонах под управлением ОС Android
\item быть удобным в использовании как для людей, владеющих музыкальной терминологией и нотной грамотой, так и для новичков
\item оценивать результаты пользователя, поощряя его за новые свершения, т.е. возможность просмотра статистики, получения “достижений” (как вариант)
\item поддерживать возможность создавать расписание занятий и получать уведомления в запланированное для них время
\item позволять использовать реальный инструмент как способ ввода информации для прохождения заданий
\item иметь разнообразные упражнения для развитиях слуха с возможностью настройки сложности 
\item поддерживать разные языки и обозначения
\end{enumerate}\par
\section{Используемые технологии}
Основными критериями выбора языка программирования были: возможность запускать код на телефонах под управлением Android, наличие удобных инструментов для мобильной разрабоки (как библиотек и фреймворков, так и IDE) и удобный синтаксис. Исходя из этих критериев, для написания приложения был выбран \textbf{Kotlin}.\par
Kotlin на данный момент является одним из самых быстроразвивающихся языков программирования\cite{Github}. Кроме этого, он исполняется в \emph{JVM} и имеет совместимость с Java-кодом, что невероятно полезно, так как за всё время их существования, накопилось огромное количество библиотек, фреймворков и прочих готовых решений для часто встречающихся проблем. Ещё одним преимуществом является поддержка Kotlin в одной из самых популярных IDE для мобильной разработки - \emph{Android Studio}, ведь она позволяет значительно ускорить все этапы разработки продукта. Наличие у Kotlin понятного и удобного синтаксиса стало решающей причиной выбрать этот язык для разработки.\par
В качестве системы сборки проекта был выбран \textbf{Gradle}, в основном из-за быстрой сборки многомодульных приложений и простоты, удобстве в настройке.\par
Фреймворк \textbf{JUnit5}, предназначенный для написания тестов под JVM, является одним их самых популярных решений. Он приятен в использовании и имеет хорошую документацию, за что и был выбран.\par
Одним из самых важных аспектов разработки ПО является создание практичного и красивого пользователю интерфейса. Для этих целей идеально подходит фреймворк \textbf{Jetpack Compose}. Он позволяет создавать интерфейс с помощью коротких конструкций кода на Kotlin, достаточно интуитивен в освоении, но при этом даёт возможность создавать современный интерфейс.
