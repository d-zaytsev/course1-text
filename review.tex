\chapter{Обзор}
\section{Анализ существующих решений}
Перед началом работы над приложением был проведён анализ существующих решений. Целью было выявление их преимуществ и недостатков, особенно важно было подчерпнуть удачные идеи, которые были бы полезны для пользователя. \par
В процессе сбора информации были рассмотрены наиболее информативные отзывы от пользователей \textit{Google Play} на несколько приложений с наибольшим числом скачиваний. Ниже приведены результаты оценки таких приложений, как \textit{Perfect Ear}(рис.\ref{fig:app1}), \textit{Functional Ear Trainer}(рис.\ref{fig:app2}), \textit{My Ear Trainer} (рис.\ref{fig:app3}).
% ---
\subsection[Perfect ear]{Perfect ear\cite{Apps1}}
\begin{minipage}[t]{0.45\textwidth}
\textbf{Достоинства:}
\begin{itemize}
  \item[+] Есть раздел с теорией, интегрированный в обучение;
  \item[+] Большое количество упражнений;
  \item[+] Оценка и поощрение достижений пользователя, дополнительная мотивация к зантиям;
  \item[+] Возможность просмотра статистики и выявления слабых мест.
\end{itemize}
\end{minipage}
\hfill
\begin{minipage}[t]{0.45\textwidth}
\textbf{Недостатки:}
\begin{itemize}
  \item[-] Возможны баги, при которых результат пользователя оценивается некорректно;
  \item[-] Интерфейс не уведомляет пользователя об ошибках;
  \item[-] Интерфейс ориентирован только на знающих терминологию людей;
  \item[-] Почти полностью отсутствует возможность настройки упражнений.
\end{itemize}
\end{minipage} 

% ---
\subsection[Functional Ear Trainer]{Functional Ear Trainer\cite{Apps2}}
\begin{minipage}[t]{0.45\textwidth}
\textbf{Достоинства:}
\begin{itemize}
  \item[+] Есть режим с полной настройкой упражнения;
  \item[+] Интерфейс адаптирован под всех пользователей (разные форматы отображения информации);
  \item[+] Большое количество настроек как интерфейса, так и всего остального.
\end{itemize}
\end{minipage}
\hfill
\begin{minipage}[t]{0.45\textwidth}
\textbf{Недостатки:}
\begin{itemize}
  \item[-] Музыкальные обозначения не подходят для всех пользователей (в России они другие);
  \item[-] Упражнения не учитывают предыдущие попытки пользователя, возможны повторения;
  \item[-] Из-за платной подписки урезан основной контент.
\end{itemize}
\end{minipage}

% ---
\subsection[My Ear Trainer]{My Ear Trainer\cite{Apps3}}
\begin{minipage}[t]{0.45\textwidth}
\textbf{Достоинства:}
\begin{itemize}
  \item[+] Уникальный режим - подбор целой мелодии на слух;
  \item[+] Большое разнообразие упражнений и выбор сложности для каждого;
  \item[+] Есть курсы с введением в теорию.
\end{itemize}
\end{minipage}
\hfill
\begin{minipage}[t]{0.45\textwidth}
\textbf{Недостатки:}
\begin{itemize}
  \item[-] Почти нет настроек интерфейса под разных пользователей;
  \item[-] Нет поддержки некоторых языков и их обозначений.
\end{itemize}
\end{minipage}

% Pictures

\begin{figure}[h]
  \centering
  \begin{subfigure}[h]{0.25\linewidth}
    \includegraphics[width=\linewidth]{app1_1.eps}
    \caption{Главный экран.}
  \end{subfigure}
  \begin{subfigure}[h]{0.25\linewidth}
    \includegraphics[width=\linewidth]{app1_2.eps}
    \caption{Упражнения.}
  \end{subfigure}
  \begin{subfigure}[h]{0.25\linewidth}
    \includegraphics[width=\linewidth]{app1_3.eps}
    \caption{Доступные задания.}
  \end{subfigure}
  \caption{Приложение Perfect Ear.}
  \label{fig:app1}
\end{figure}

\begin{figure}[h]
  \centering
  \begin{subfigure}[h]{0.25\linewidth}
    \includegraphics[width=\linewidth]{app2_1.eps}
    \caption{Настройка упражнения.}
  \end{subfigure}
  \begin{subfigure}[h]{0.25\linewidth}
    \includegraphics[width=\linewidth]{app2_2.eps}
    \caption{Статистика.}
  \end{subfigure}
  \begin{subfigure}[h]{0.25\linewidth}
    \includegraphics[width=\linewidth]{app2_3.eps}
    \caption{Доступные задания.}
  \end{subfigure}
  \caption{Приложение Functional Ear Trainer.}
  \label{fig:app2}
\end{figure}

\begin{figure}[h]
  \centering
  \begin{subfigure}[h]{0.25\linewidth}
    \includegraphics[width=\linewidth]{app3_1.eps}
    \caption{Настройка упражнения.}
  \end{subfigure}
  \begin{subfigure}[h]{0.25\linewidth}
    \includegraphics[width=\linewidth]{app3_2.eps}
    \caption{Задание повышенной сложности.}
  \end{subfigure}
  \begin{subfigure}[h]{0.25\linewidth}
    \includegraphics[width=\linewidth]{app3_3.eps}
    \caption{Выбор задания.}
  \end{subfigure}
  \caption{Приложение Functional Ear Trainer.}
  \label{fig:app3}
\end{figure}

\subsection{Результаты анализа}
Исходя из приведённого выше обзора, можно сделать вывод, что на данный момент уже существует несколько решений, предоставляющих пользователям возможность обучаться музыке, используя современные технологии. Это говорит о присутствии реальной потребности в приложениях, помогающих в освоении музыкальных навыков.\par
В то же время, ни одно из них полностью не решает проблемы. Почти в каждом приложении основной контент доступен лишь по подписке, интерфейс не адаптирован под пользователей с разными знаниями, иногда допускаются грубые ошибки при подборе или проверке заданий, доступно очень маленькое количество настроек как пользовательского интерфейса, так и самих упражнений. Более того, для русских пользователей, желающих учиться по принятым в нашей стране обозначениям, круг выбора сводится всего до нескольких вариантов.\par
Однако, при разработке представленных аналогов было реализовано множество хороших решений, которые следовало бы особо отметить: большое разнообразие различных упражнений с разделением на уровни сложности, есть возможность осваивать теорию (словари с терминами, статьи) и параллельно закреплять знания на практике, почти во всех приложениях ведётся анализ достижений пользователя и сбор статистики. Также одна из интересных функций -- режим, где возможна полная настройка занятия (представлено лишь в \textit{Functional Ear Trainer}, но по отзывам многим это нужно).
\section{Требования к приложению}
Исходя из результатов проведённого анализа, были сформулированы требования. Обучающее приложение должно: 
\begin{enumerate}
\item запускаться на устройствах под управлением ОС Android;
\item быть удобным в использовании как для людей, владеющих музыкальной терминологией и нотной грамотой, так и для новичков;
\item оценивать результаты пользователя, поощряя его за новые свершения, например возможность просмотра статистики, получения “достижений” (как вариант);
\item поддерживать возможность создавать расписание занятий и получать уведомления в запланированное для них время;
\item позволять использовать реальный инструмент как способ ввода информации для прохождения заданий;
\item иметь разнообразные упражнения для развитиях слуха с возможностью настройки сложности;
\item поддерживать разные языки и обозначения.
\end{enumerate}\par
\section{Используемые технологии}
Основными критериями выбора языка программирования были: возможность запускать код на телефонах под управлением Android, наличие удобных инструментов для мобильной разрабоки (как библиотек и платформ, так и IDE) и удобный синтаксис. Исходя из этих критериев, для написания приложения был выбран \textbf{Kotlin}.\par
Kotlin на данный момент является одним из самых быстроразвивающихся языков программирования\cite{Github}. Кроме этого, он исполняется в \emph{JVM} и имеет совместимость с Java-кодом, что невероятно полезно, так как за всё время их существования, накопилось огромное количество библиотек, платформ и прочих готовых решений для часто встречающихся проблем. Ещё одним преимуществом является поддержка Kotlin в одной из самых популярных IDE для мобильной разработки - \emph{Android Studio}, ведь она позволяет значительно ускорить все этапы разработки продукта. Наличие у Kotlin понятного и удобного синтаксиса стало решающей причиной выбрать этот язык для разработки.\par
В качестве системы сборки проекта был выбран \textbf{Gradle}, в основном из-за быстрой сборки многомодульных приложений и удобстве в настройке.\par
Фреймворк \textbf{JUnit5}, предназначенный для написания тестов под JVM, является одним их самых популярных решений. Он приятен в использовании и имеет хорошую документацию, за что и был выбран.\par
Одним из самых важных аспектов разработки ПО является создание практичного и красивого пользователю интерфейса. Для этих целей идеально подходит платформа \textbf{Jetpack Compose}. Он позволяет создавать интерфейс с помощью коротких конструкций кода на Kotlin, имеет низкий порог вхождения, но при этом даёт возможность создавать современный интерфейс.
