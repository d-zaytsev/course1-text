\chapter{Введение}
\setlength\parindent{1.5em} 
\par 
Музыка всегда была одним из самых популярных видов искусств. Способность выражать мысли и чувства с помощью звука никогда не переставала интересовать человечество. Множество людей с самыми разными целями и подходами к обучению выбирали и продолжают выбирать музыку как объект для своего изучения. В наше время интерес к ней всё так же силён\cite{thamprasert2023network}.

Кроме того, по результатам проведённых исследований\cite{dumont2017music}, можно сделать вывод о том, что обучение музыке с ранних лет жизни способствует развитию у ребёнка:
\begin{itemize}
\item Памяти
\item Моторики
\item Внимания
\item Навыков коммуникации
\item Способности различать сложные эмоции и контролировать их
\item Навыков владения языком и речью
\end{itemize}\par
Что, безусловно, говорит о полезности прививания любви к данному виду творчества.\par

Изучение музыки - очень трудоёмкий процесс, требующий от человека не только желания погрузиться в новую область знаний, но и регулярных занятий, постоянной работы над ошибками, наличия рядом опытного преподавателя. Одним из важнейших элементов обучения является развитие \textit{музыкального слуха}, т.е. способности анализировать музыку без прямого чтения нот. Этот навык необходим как ученикам музыкальных школ для успешной сдачи экзаменов, так и людям, изучающим музыку самостоятельно.\par

Однако, несмотря на стремительное развитие технологий, на данный момент существует не так много эффективных решений для развитиях этих навыков, объединяющих в себе удобство и практичность.